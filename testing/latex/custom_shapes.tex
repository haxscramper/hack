\documentclass[tikz, border=1mm]{article}
\usepackage{tikz}
\usepackage{xparse}

\usetikzlibrary{calc,arrows}
\usetikzlibrary{circuits.logic.IEC}
\usetikzlibrary{math}

\tikzset{
  usingPic/.pic={
    \draw (0,0) rectangle (2,2);
    \draw [ultra thick, blue] (0,1) -- (-1,1) node {} coordinate (-input);
    \draw [ultra thick, red] (2,1) -- (3,1) node {} coordinate (-output);
  }
}

\newcommand{\usingNewc}[2]{
  \begin{scope}[shift={(#1)}] % Shift origin position
    \draw (0,0) rectangle (2,2);
    \draw [ultra thick, blue] (0,1) -- (-1,1) node {}
      coordinate (#2 -input); % Set node name
    \draw [ultra thick, red] (2,1) -- (3,1) node {}
      coordinate (#2-output);
  \end{scope}
}


\begin{document}

Pic created using tikzset

\begin{tikzpicture}
  \pic (dff) at (6, 0) {usingPic};
  \pic (dfff) at (1, 0) {usingPic};

  \draw [ultra thick, green] (dff-input) -- (dfff-output);
\end{tikzpicture}

Pic created using macro

\begin{tikzpicture}
  \usingNewc{6, 0}{dff};
  \usingNewc{1, 0}{dfff};

  \draw [ultra thick, green] (dff -input) -- (dfff-output);
\end{tikzpicture}

\end{document}
